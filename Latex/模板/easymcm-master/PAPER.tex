%%%%%%%%%%%%%%%%%%%%%%%%%%%%%%
% 	   美赛模板,正文部分		 
%          PAPER.tex         
%%%%%%%%%%%%%%%%%%%%%%%%%%%%%%

\documentclass[12pt]{article}

% 请在此填写控制号、题号和标题,年份不需要填(自动以当前电脑时间年份为准)
\usepackage[2012050]{easymcm}\problem{A}   
\usepackage{palatino} % 这个是COMAP官方杂志采用的字体,如不需要可注释掉,以使用默认字体
\title{An MCM Paper Made by Team 1234567}  % 标题

% 如您参加的是ICM(即选择了D/E/F题),请使用以下的命令修改Summary Sheet题头
% \renewcommand{\contest}{Interdisciplinary Contest in Modeling (ICM) Summary Sheet}

% 正文开始
\begin{document}
\begin{abstract} 
\setlength{\parskip}{1em}
E-commerce is growing at an unprecedented rate all over the world. In the process of purchasing goods, ratings and reviews play a vital reference role. Companies are pursuing a comprehensive and understandable analysis of online market data to craft greater success.

In this paper, we seek to devise several approaches to analyze the product evaluation by exploring ratings, text-based reviews and other related indicators.

We first perform exploratory data analysis by generating the data quality report and studying the distribution of the most critical indicators. Then we preprocess reviews by a series of steps, including removing punctuations, converting abbreviations, etc. Besides, we extract the frequent features of each product using \textbf{WordCloud}.

We then build PRMP, a framework that defines the patterns, relationships, measures, and parameters within and between ratings and reviews. We use \textbf{SentiWordNet} to obtain the sentiment of a review and normalize star ratings and helpfulness. Through \textbf{association analysis}, we visualize the relationship using a set of heat maps and draw conclusions from them.  

After that, we propose a new approach to find a traceable measure for the product. We use \textbf{Entropy Weight Method} to obtain weights of the indicators. To identify reputation trends over time, we use the \textbf{ARIMA Model} to fit the reputation score. We select one of the hair dryer products as our main study object, calculate its score over time and give its most likely trending result.  We use the \textbf{Non-linear Programming Model} to find the best combination of text-based and rating-based measures to indicate potential success and failure. We apply the \textbf{Hovland Persuasion Model} to build a decision model that describes the indicators that influence customer decisions, achieving the combination of the theory of social psychology and real-life context. And to analyze specific quality descriptors, we classify words into eight categories using the \textbf{NRC Emotion Lexicon}. Then we match the emotional intensity with star ratings and find the relationship between them.

Finally, based on the established model and detailed analysis, we put forward practical suggestions for Sunshine Company's marketing plan to improve its product competitiveness.

	
	
	% 美赛论文中无需注明关键字。若您一定要使用,
	% 请将以下两行的注释号 '%' 去除,以使其生效
	% \vspace{5pt}
	% \textbf{Keywords}: MATLAB, mathematics, LaTeX.

\end{abstract}



		% 摘要请到ABSTRACT.tex中填写

\section{Introduction}
\subsection{Problem Background}
Here is the problem background ...

Two major problems are discussed in this paper, which are:
\begin{itemize}
    \item Doing the first thing.
    \item Doing the second thing.
\end{itemize}

\subsection{Literature Review}
A literatrue\cite{1} say something about this problem ...

\subsection{Our work}
We do such things ...

\begin{enumerate}[\bfseries 1.]
    \item We do ...
    \item We do ...
    \item We do ...
\end{enumerate}

\section{Preparation of the Models}
\subsection{Assumptions}

\subsection{Notations}
The primary notations used in this paper are listed in \textbf{Table \ref{tb:notation}}.
\begin{table}[!htbp]
\begin{center}
\caption{Notations}
\begin{tabular}{cl}
	\toprule
	\multicolumn{1}{m{3cm}}{\centering Symbol}
	&\multicolumn{1}{m{8cm}}{\centering Definition}\\
	\midrule
	$A$&the first one\\
	$b$&the second one\\
	$\alpha$ &the last one\\
	\bottomrule
\end{tabular}\label{tb:notation}
\end{center}
\end{table}

\section{The Models}
\subsection{Model 1}
\subsubsection{Detail 1 about Model 1}
\begin{equation}
    e^{i\theta}=\cos\theta+i\sin\theta.
\end{equation}

\section{Strengths and Weaknesses}
\subsection{Strengths}
\begin{itemize}
    \item First one...
    \item Second one ...
\end{itemize}

\subsection{Weaknesses}
\begin{itemize}
    \item Only one ...
 \end{itemize}

\begin{thebibliography}{99}
\addcontentsline{toc}{section}{References}  %引用部分标题("Refenrence")的重命名
\bibitem{1}Elisa T. Lee, Oscar T. Survival Analysis in Public Health Research. \emph{Go. College of Public Health}, 1997(18):105-134.
\bibitem{2}Wikipedia: Proportional hazards model. 2017.11.26. \texttt{\\https://en.wikipedia.org/wiki/Proportional\_{}hazards\_{}model}
\end{thebibliography}


% ==============以下为附录内容,如您的论文中不需要程序附录请自行删除====================
\clearpage
\begin{subappendices}						% 附录环境
\section*{Apendix: The source codes}		% 附录标题可以自行修改
\addcontentsline{toc}{section}{Appendix}  	% 将附录内容加入到目录中

This MATLAB program is used to calculate the value of variable $a$.
\begin{lstlisting}[language=Matlab, caption=\texttt{temp.m}]
a = 0;
for i = 1:5
	a = a + 1;
end
\end{lstlisting}

This LINGO program is used to search the optimize solution of 0-1 problem.
\begin{lstlisting}[language=Lingo, caption=\texttt{temp.lg4}]
model:
sets:
WP/1..12/: M, W, X;
endsets
data:
M = 2 5 18 3 2 5 10 4 11 7 14 6;
W = 5 10 13 4 3 11 13 10 8 16 7 4;
enddata
max = @sum(WP:W*X);
@sum(WP: M * X)<=46;
@for(WP: @bin(X));
end
\end{lstlisting}

\end{subappendices}
% =================================================================================



\end{document}